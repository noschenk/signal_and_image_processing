\documentclass[a4paper,8pt]{extarticle}
\usepackage[]{graphicx}
\usepackage[]{color}
\usepackage[english]{babel}  % force American English hyphenation patterns
\usepackage[utf8]{inputenc}  % unicode
% \usepackage{graphicx}
% \usepackage{lipsum}  % generate text
\usepackage{xcolor}  % have more than basic colors
\definecolor{mygray}{rgb}{0.8,0.8,0.8}
\usepackage{hyperref}
\usepackage{listings}  % include code, code style
\lstset{basicstyle=\ttfamily, breaklines = true, backgroundcolor=\color{mygray},language=bash,basicstyle=\ttfamily,columns=fullflexible}
\usepackage{wrapfig}  % figures can be wrapped in text
% \usepackage{mathtools}  % have math symbols
\usepackage{amsmath}
\usepackage{amssymb}
\usepackage{booktabs}  % to make professional tables
% ------------------------------------------------------------------------- document descriptions
\title{Signal and Image Processing - HW3}
\author{Noëlle Schenk}
\date{\today, time invested : 9.5h}
\begin{document}
\maketitle
% \tableofcontents
% \newpage
\section{Comments}
none

\section{Exercise 1}
\begin{figure}
  \caption{horizontal line example.}
  \centering
    \includegraphics[width=0.6\textwidth]{ex1_1_horizontal_line_example.png}
  \label{fig:horizontal}
  \caption{Line plot for a line that has been added minimal noise to not be perfectly horizontal any more.}
\end{figure}

\subsection{Exercise 1.1}
Adding minimal noise for vertical and horizontal lines gives almost horizontal and almost vertical lines. \texttt{sys.float\_info.epsilon} gives the smallest floating point number that is possible in python. A possible horizontal line plot can look like the one in Figure \ref{fig:horizontal}. 

\subsection{Exercise 1.2}
The distance from a point $(x_0, y_0)$ to a line $y = mx + b \Leftrightarrow mx + (-1)y + b = 0$ is given by \[ d = \frac{| mx_0 + (-1) y_0 + b |}{\sqrt{m^2 + (-1)^2}} = \frac{|mx_0 - y_0 + b |}{m} \]



\subsection{Exercise 1.3}

\begin{figure}
  \caption{Choice of sigma.}
  \centering
    \includegraphics[width=0.32\textwidth]{ex1_bridge_sigma1.png}
    \includegraphics[width=0.32\textwidth]{ex1_pool_sigma1.png}
    \includegraphics[width=0.32\textwidth]{ex1_tennis_sigma1.png}
    
    \includegraphics[width=0.32\textwidth]{ex1_bridge_sigma6.png}
    \includegraphics[width=0.32\textwidth]{ex1_pool_sigma6.png}
    \includegraphics[width=0.32\textwidth]{ex1_tennis_sigma6.png}

  \label{fig:sigma}
  \caption{Figures above: sigma = 1, figures below: sigma = 6. The smoothing parameter sigma defines how much noise is left in the image. Its choice depends on our images - if the line to detect is well contrasted, we can put sigma higher and smooth out some noise. If the line is less well detectable, sigma must be smaller. Here, with a sigma of 1 we would have much too much noise. We only search the best fitting line, reducing noise is therefore required. With sigma = 6, we still see the most dominant lines in the images while much noise is smoothed out. It is the highest possible value where the dominant lines are still visible.}
\end{figure}
See figure \ref{fig:sigma}.


\subsection{Exercise 1.5}

\begin{figure}
  \caption{RANSAC.}
  \centering
    \includegraphics[width=0.32\textwidth]{ex1_5_edgemap1.png}
    \includegraphics[width=0.32\textwidth]{ex1_5_lines.png}
    
    \includegraphics[width=0.32\textwidth]{ex1_5_edgemap2.png}   
    \includegraphics[width=0.32\textwidth]{ex1_5_lines2.png}
    \includegraphics[width=0.32\textwidth]{ex1_5_edgemap22.png}

    \includegraphics[width=0.32\textwidth]{ex1_5_edgemap3.png}   
    \includegraphics[width=0.32\textwidth]{ex1_5_edgemap31.png}
    \includegraphics[width=0.32\textwidth]{ex1_5_edgemap34.png}
    
  \label{fig:ransac}
  \caption{\textbf{RANSAC with 500 iterations:} Above : RANSAC on the sample image 'synthetic'. The noise added to the image is almost completely removed by smoothing and the line can be detected very well. Middle : RANSAC on the image 'bridge'. As ransac randomly samples points from an image, the perfect line can not always be detected. Either the code must be run more than once or the number of iteration should be increased. The same problem is also shown below with the sample image 'pool'. The shown issue can happen with all images (including 'synthetic'.}
\end{figure}
RANSAC performed on example images, see figure \ref{fig:ransac}.


\section{Exercise 2}

\begin{figure}
  \centering
    \includegraphics[trim={110 260 110 260}, clip, width=\textwidth]{ex2_tex_synt_donkey.png}

    \includegraphics[trim={90 210 90 210}, clip, width=\textwidth]{ex2_tex_synt_tomato.png}

    \includegraphics[trim={120 270 120 270}, clip, width=\textwidth]{ex2_tex_synt_sea.png}
    
  \label{fig:texsynt}
  \caption{\textbf{Texture synthesis} with the same algorithm on 3 different pictures.}
\end{figure}

The results are shown in figure \ref{fig:texsynt}. The algorithm produces results for all three input pictures, however not at the same precision as for the donkey picture. \\
A main problem of my code was the color: the mask was generated from the partly filled image. As image.shape was x,x,3 - I chose one color channel only for the mask. For the green picture, for example, the best color channel to pick would be 1 (green). The picture contains many green values, so the overall intensity is high. Less or even no pixels form the original image are entirely black in this case. However with a lower intensity color channel the mask can get noise when black pixels in the shadow are interpreted as mask.\\
From the texture point of view, the grass and ocean were recovered best, the tomato results are less natural. This is due to the fact that the patch size is too small to capture the entire repetitive elements of the tomato picture. Also, it would make a difference if only the middle pixel would be replaced in the image to fill.

% or run all following commands (are in runlatex.sh)
% KNITR
% pdflatex lab_knitr.tex
% pdflatex lab_knitr.tex
% biber lab_knitr
% pdflatex lab_knitr
% gnome-open lab_knitr.pdf
\end{document}
